\documentclass[english]{scrartcl}

\title{Analytical Methods for the Riyadh Multiplex}
\subtitle{v0.0}
\author{Zeyad Al-awwad, Shan Jiang, and Phil Chodrow}
\date{October 15, 2015}

\usepackage[T1]{fontenc}
\usepackage[utf8]{inputenc}
\usepackage{babel}
\usepackage{blindtext}
\usepackage{amsmath}
\usepackage{amsthm}
\usepackage{amsfonts}
\usepackage{amssymb}

\setkomafont{disposition}{\normalfont\bfseries}

\newtheorem{thm}{Theorem}
\newtheorem{lm}{Lemma}
\newtheorem{cor}{Corrolary}
\newtheorem{clm}{Claim}
\newtheorem*{thm*}{Theorem}
\newtheorem*{lm*}{Lemma}
\newtheorem*{cor*}{Corrolary}
\newtheorem*{clm*}{Claim}

\newcommand\abs[1]{\left|#1\right|}
\newcommand\E[0]{\mathbb{E}}
\newcommand\R[0]{\mathbb{R}}


\begin{document}

\maketitle

\section{Purpose}

	The purpose of this document is to outline the analytical methods to be used in MIT and CCES's article analyzing the Riyadh Metro proposal from the standpoint of multiplex network theory. This document defines a notational framework, formulates some key quantities, and illustrates how these quantities can be used to answer natural questions about the Riyadh multiplex. It does not define a software-level approach to performing the analysis, but may inform such an approach. 

	After many revisions, this document may become the foundation of the `Methods' section of the final paper, or a technical appendix. 

\section{Mathematical Framework}

	Let $N$ be a weighted directed network with vertex set $V$ and edge set $E$. We say that $N$ is a \emph{multiplex composed of layers $N_i$, $i = 1,\ldots,n$ } under the following conditions:
	\begin{enumerate}
		\item \emph{Layers are subnetworks}: each $N_i$ is a subnetwork of $N$
		\item \emph{Layers partition vertices:} $V_i \cap V_j = \emptyset$ if $i \neq j$ and  $\bigcup_i V_i = V$.
		\item \emph{Layer edges are disjoint:} $E_i \cap E_j = \emptyset$ if $i \neq j$
	\end{enumerate}
	We define a \emph{transfer} between layers to be an edge $e\in E$ such that $e\notin E_i$ for any $i$. Every such edge must indeed connect two layers, since the vertices must each lie in distinct layers. If these layers are $i$ and $j$, then write $e\in E_{ij}$. Let $N_{ij}$ denote the network determined by these edges and the vertices they connect. We refer to $N_{ij}$ as the \emph{transfer network} between layers $i$ and $j$. 

	Let $I$ be an index set with elements in $1,\ldots, n$. Then, let $N_I$ be the subnetwork of $N$ consisting of the layers $I$ and all transfers between them. Formally, 
	\begin{enumerate}
		\item $V_I = \bigcup_{i\in I} V_i$
		\item $E_I = \bigg(\bigcup_{i\in I} E_i\bigg) \cup \bigg(\bigcup_{i,j\in I} E_{ij}\bigg)$
	\end{enumerate}

	Three quantities are of primary importance to our analysis:
	\begin{enumerate}
		\item The cost of shortest paths between vertices
		\item The weighted betweenness centrality of a vertex
		\item The set of vertices that can be reached from a given vertex for fixed cost. 
	\end{enumerate}

	\subsection{Cost of Shortest Paths}

		Let $\delta:E\rightarrow \R$ be a nonnegative weight function on each edge of $N$. Let $p$ be a path between nodes $u$ and $v$ through the network $N_I$. We define $\delta(p) = \sum_{e \in p} \delta(e)$. Write $p^{I}(u,v)$ for the path between $u$ and $v$ that minimizes $\delta(p)$ among all such paths through $N_I$, and write $c^{I}(u,v)$ for its cost $\delta(p^{I}(u,v))$. Here and forward, we assume that $u,v \in N_I$, as otherwise there is no path between them through $N_I$. (\emph{NB: does it matter if these paths are not unique? In our application, they almost always will be.}). 

	\subsection{Weighted Betweenness}
		We say that a vertex $u$ is \emph{between} $v$ and $w$ if $u$ lies on a shortest path $p^{I}(v,w)$, and similarly for an edge $e$. Let $g^{I}_{v,w}(u)$  be the indicator function for this event. Let $d_{v,w}$ be a set of nonnegative scalar node weights and $D = \sum_{v,w \in V} d_{v,w}$; then, we define the weighted betweenness centrality of vertex $u$ through $N_I$ by $d$ as 
		\begin{equation}
			b^{I}(d;u) =   \frac{1}{D} \sum_{v,w \in V_I} d_{v,w}g^{I}_{v,w}(u)
		\end{equation} 
		We should distinguish two senses of the word 'weight': the metric $\delta$ assigns costs to edges, while the weights $d_{v,w}$ assigns relative importance to nodes in the betweenness calculation. 

		It may also be useful to work with the normalized betweenness centrality given by $\bar{b}^{I}(d;u) = \frac{b^{I}(d;u)}{(\abs{V_I} - 1)(\abs{V_I - 2)})}$.
	\subsection{Outreach}
		Fix $c > 0$. The \emph{outreach set of vertex $u$ for cost $c$} is the set of nodes reachable from $u$ for cost no greater than $c$:

		\begin{equation*}
			r^{I}(c;u) = \{ v : c^{I}(u,v) \leq c \}
		\end{equation*}

\section{Application to Riyadh Multiplex}
	The purpose of this section is to illustrate how natural questions for the Riyadh network can be computed using this framework. Throughout this section, we'll use the following networks.
	\begin{itemize}
		\item $W$, the streets regarded as a walkable network. 
		\item $D$, the streets regarded as a drivable network. This network is a copy of $W$ but with different travel time costs; i.e. $\delta$ is different on the two networks. 
		\item $T$, the set of TAZ connector nodes. This is copy of a subset of $W$. We assume that every element of $T$ shares an edge with the corresponding element of $W$ of zero cost. There are no edges between nodes in $T$.  
		\item $M$, the metro network.
		\item $B$, the bus network. This is a copy of a subset of $D$, with the same travel time costs. 
	\end{itemize}

	\subsection{Mobility}
		\begin{itemize}
			\item \textbf{Question:} How will the introduction of the metro enhance overall mobility for non-car owners in Riyadh? \textbf{Approach}: Let $I = \{W, B\}$ and $J = \{W, B, M\}$. Compare spatial outreach sets $r^{I}(c;u)$ and $r^{J}(c;u)$ for different values of $c$. An appropriate metric might be the area of the convex hull of the spatial outreach set; another might be the mean of the radii of the nodes comprising the convex hull. 
			\item \textbf{Question}: How will the introduction of the metro reduce commute time for non-car owners in Riyadh? \textbf{Approach}: Let $I$ and $J$ be as above, and compute $\sum_{u,v} d_{u,v} c^{I}(u,v)$, where $d_{u,v}$ is the travel demand from vertex $u$ to vertex $v$. Compare to $\sum_{u,v} d_{u,v} c^{J}(u,v)$. \textbf{Alternative approach}: The previous approach is abstract, and doesn't take into account that we only have travel demand between OD pairs, i.e. elements of $T$. Let $I = \{ T, W, B \}$ and $J = \{T, W, B, M \}$. Then, we can compute mean commute time without the metro as 
			$$t^I = \frac{1}{D \abs{T}(\abs{T}-1)}\sum_{u,v \in T} d_{u,v} c^{I}(u,v),$$ 
			where $u$ and $v$ range over the set of TAZ connector vertices. We can then compute $t^J$, mean commute time with the metro, by replacing $I$ with $J$ in the expression above. 
			\item \textbf{Question:} How will the introduction of the metro enhance mobility for women in Riyadh? \textbf{Approach:} Let $w_u$ denote the number of women at node $u$ and $W$ the total number of women recorded in our data set. We can then compute (a) the mean spatial radius of the sets $r^I(c;u)$ weighted by $w_u$. We can also compute (b) the mean commute time without the metro as  
			$$t^I = \frac{1}{W D\abs{T}(\abs{T}-1)}\sum_{u,v \in T} w_u d_{u,v} c^{I}(u,v),$$ 
			and similarly for $J$. Note that this approach assumes that overall travel demand patterns are the same for women as for men. 
		\end{itemize}

	\subsection{Congestion}
		\begin{itemize}
			\item \textbf{Question:} How will the introduction of the metro impact walking congestion throughout the city? \textbf{Approach}: Let $I = \{ T, W, B \}$ and $J = \{T, W, B, M \}$. We define the following restricted betweenness measure
			\begin{equation*}
				b^{I}_T(d;u) = \frac{1}{D} \sum_{v,w \in T} d_{v,w}g^{I}_{v,w}(u),
			\end{equation*} 
			which computes a betweenness score for every node in $N_I$ based only on demand information in $T$. 
			Compare the distributions of $b_T^{I}(v)$ and $b_T^{J}(v)$.
			\item \textbf{Question:} Assume that drivers will adopt the metro with some probability if they can reach their workplace more quickly through it. How will street congestion be impacted? \textbf{Approach:} Let $I = \{T, W, B, M \}$ and $J = \{ T, D \}$. Compare $c^{I}(u,v)$ to $c^{J}(u,v)$ for each $u,v$ to estimate the reduced number of drivers in each OD pair. Run through Shan's congestion model and compare to baseline. Also, consider computing the estimated reduced person-hours on the network. This approach is consistent with applying different probabilities of adoption to different demographic segments. 
		\end{itemize}

	\subsection{Mobility/Congestion Interactions}
		\begin{itemize}
			\item \textbf{Question:} Is there a tradeoff between increased mobility and new congestion? E.g. living near a subway stop is convenient, but noisy. \textbf{Approach:} Let $I = \{ T, W, B \}$ and $J = \{T, W, B, M \}$. Compute $\Delta r_c(v) = r^{J}(c;v) - r^{I}(c;v)$ for fixed $c$, compute $\Delta b(v) = b^{J}(v) - b^{I}(v)$, and test for correlation.
			\item \textbf{Question:} Is there a tradeoff between decreased commute times and new congestion?  \textbf{Approach:} Let $I = \{ T, W, B \}$ and $J = \{T, W, B, M \}$. Compute the change in mean travel time as a above, compute $\Delta b(v) = b^{J}(v) - b^{I}(v)$, and test for correlation.
		\end{itemize}

\end{document} 

 